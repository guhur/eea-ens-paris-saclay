\documentclass[conference]{IEEEtran}
\newcommand*{\rootPath}{../}
\usepackage{url}
\usepackage[utf8]{inputenc}
\usepackage{tabulary}
\usepackage{tabularx}
\usepackage{listings} % code highlighting
\lstset{language=Ruby,
                basicstyle=\ttfamily,
                keywordstyle=\color{blue}\ttfamily,
                stringstyle=\color{red}\ttfamily,
                commentstyle=\color{green}\ttfamily,
                morecomment=[l][\color{magenta}]{\#}
}


% graph
\usepackage{graphicx}
\usepackage[outdir=./]{epstopdf}
\usepackage{array}



% \usepackage[french]{babel}


\standalonetrue

\begin{document}

%%=============================================================================



%%=============================================================================
%-------------------------------------------------------------------------------
%   MOUVEMENT
%-------------------------------------------------------------------------------

\section{Recopie du mouvement d'un bras}
\label{sec:intro}

\subsection{Présentation du projet}
Tandis que la partie précédente revisitait les concepts du langage C++, on propose désormais d'écrire un nouveau projet pour le framework \darwin.
Ce dernier va recopier les mouvements imposés au bras droit sur le gras gauche.

Pour faciliter la programmation, on différenciera les fichiers suivants:
\begin{itemize}
    \item Arm.cpp et Arm.h formeront un MotionModule, appelé Arm,
    pour proposer une interface de gestion d'un bras. On placera ces fichiers dans \inlinecode{Framework/src/motion/modules} et \inlinecode{Framework/include}.
    \item  \inlinecode{main.cpp}, situé dans
    \inlinecode{\url{Linux/project/recopier_bras}},
    lancera le programme et éxecutera les méthodes de Arm.
\end{itemize}

\subsection{Classe Arm}

La classe Arm contiendra les attributs et méthodes suivantes :
\begin{itemize}
    % \item GetLeftArm permettra de choisir ces bras ;
    \item AddMotors ;
    \item GetMotorsValues retournera les valeurs des servo-moteurs du bras ;
    \item Scan retournera un tableau avec les positions sur les servo-moteurs ;
    \item Apply prendra en argument un tableau d'ordres sur les servo-moteurs de hardArm et appliquera ces ordres ;
    \item PutSoft et PutHard rendront respectivement le bras mou (on peut imposer manuellement un mouvement) ou dur (on ne peut pas).
\end{itemize}

\Que{Tracez le diagramme de classes de Arm et des objets auxquels Arm se réfère.}

\Ans{\todo{à faire}}

\Que{Programmez \inlinecode{Arm.cpp} et \inlinecode{Arm.h}. Aucun test ne sera pour le moment effectué.}

\Ans{cf. répertoire git.}


\subsection{Projet recopier\_bras}

Ainsi, le fichier main.cpp suivra l'algorithme suivant :
\begin{enumerate}
    \item initialiser le module Arm pour les bras softArm (celui dont on impose manuellement le mouvement) et hardArm (celui qui recopie les mouvements) ;
    \item faire un boucle infini avec :
    \item enregistrer les informations sur softArm ;
    \item appliquer les valeurs sur hardArm.
\end{enumerate}

\Que{Réalisez ce programme. On ne demande pas de compiler le fichier}
\Ans{cf. répertoire git.}

\subsection{Compilation}

Pour compiler le code, on partira d'un Makefile pré-existant, par exemple celui contenu dans \inlinecode{\url{Linux/project/walk_tuner}}.

\Que{Adaptez le Makefile puis lancer la compilation du projet en lançant \inlinecode{make} depuis le dossier du projet. Corrigez d'éventuels erreurs sur le programme puis testez le.}
\Ans{cf. répertoire git.}


\end{document}
