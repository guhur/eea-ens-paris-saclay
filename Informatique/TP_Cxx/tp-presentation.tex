% \documentclass[runningheads,a4paper]{llncs}
\documentclass[abstracton]{scrartcl}
% \documentclass{exam}
\newcommand*{\rootPath}{./}

\usepackage{url}
\usepackage[utf8]{inputenc}
\newcommand{\inlinecode}{\texttt}
\usepackage{microtype}
% \usepackage[colorlinks]{hyperref}

% math and cs
% \usepackage[lined,boxed,commentsnumbered]{algorithm2e}
% \usepackage{amsmath}
% \usepackage{amssymb}
% \usepackage{mathrsfs}

% style
% \usepackage{booktabs}
% \usepackage{multirow}
\usepackage{listings} % code highlighting
\lstset{language=C++,
                basicstyle=\ttfamily,
                keywordstyle=\color{blue}\ttfamily,
                stringstyle=\color{red}\ttfamily,
                commentstyle=\color{green}\ttfamily,
                morecomment=[l][\color{magenta}]{\#}
}
\usepackage{todonotes}
\usepackage{standalone}


% graph
\usepackage{graphicx}
\usepackage[outdir=./]{epstopdf}
\usepackage{array}



% \usepackage[french]{babel}
\usepackage[nomain,acronym,xindy,toc]{glossaries}
\makeglossaries
\usepackage[xindy]{imakeidx}
\makeindex




% questions
\newcounter{question}
\newcommand\Que[1]{%
    \bfseries
   \leavevmode\par
   \stepcounter{question}
   \noindent
   Q  --- #1\par\mdseries}


% reponses
\def\showanswers{1}% Set this =0 to hide, =1 to show
\newcommand\Ans[2][]{%
    \ifnum\showanswers=1
        \bfseries
        \leavevmode\par\noindent
       {\leftskip37pt
        R --- \textbf{#1}#2\par}\mdseries
    \fi
}


\newcommand\darwin{\texttt darwin-op}











\newglossaryentry{firmware}{
    name={firmware},
    description={programme intégré dans un matériel informatique (ordinateur, photocopieur, automate (API, APS), disque dur, routeur, appareil photo numérique, etc.) pour qu'il puisse fonctionne}
}


\newglossaryentry{framework}{
    name={framework},
    description={ensemble cohérent de composants logiciels structurels, qui sert à créer les fondations ainsi que les grandes lignes de tout ou d’une partie d'un logiciel (architecture)}
}
\newglossaryentry{instance}{
    name={instance},
    description={objet constituant un exemplaire de la classe}
}

\newcolumntype{L}[1]{>{\raggedright\let\newline\\\arraybackslash\hspace{0pt}}m{#1}}
\newcolumntype{C}[1]{>{\centering\let\newline\\\arraybackslash\hspace{0pt}}m{#1}}
\newcolumntype{R}[1]{>{\raggedleft\let\newline\\\arraybackslash\hspace{0pt}}m{#1}}

\standalonetrue

\title{Séries de travaux pratiques en C++}


\author{Pierre-Louis Guhur,      Thomas Rodet}



\makeglossaries

\begin{document}





\maketitle

\begin{abstract}
C++ est un langage de programmation qui est compilé, avec une large variété de plateformes le supportant.
Il implémente différentes paradigmes, comme la programmation procédurale  la programmation orientée objet, et la programmation générique.
C'est l'un des langages les plus populaires, et est disponible sur la plupart des plateformes.
En particulier, il est utilisé pour coder le robot Darwin-Op.
L'objectif de cette série de travaux pratiques sur le C++ est de visiter certains concepts propres à ce langage et de les expérimenter sur le robot.
Deux mini-projets sont proposées : le premier réalise une liste chaînée, et le second un module de mouvement pour le robot.
\end{abstract}


\section{Présentation des différents travaux pratiques}
\label{sec:presentation}

\begin{enumerate}
    \item à partir des listes chaînées, découvrir la programmation orientée objet
    \item amélioration du code des listes chaînées en intégrant des templates de classes
    \item familiarisation avec le robot : analyse du diagramme des classes et implémentation d'un module de mouvement pour gérer le mouvement d'un bras
    \item développement d'un projet pour recopier la position d'un bras imposé sur l'autre bras.
\end{enumerate}


\section{Volume horaire}
\label{sec:heures}

\section{Inscription dans une séquence pédagogique}
\label{sec:sequence}




% \section{Conclusion}
% \label{sec:ccl}

%
% Ce TP a permis d'explorer le langage C++ et de se familiariser avec le robot Darwin-Op. En particulier, les concepts suivants ont été vus :
% \begin{itemize}
%     \item utilisation d'un framework au-dessus d'un système d'exploitation pour programmer un robot (ce qui n'est pas sans rappelé ROS) ;
%     \item programmation orienté objet et générique avec les concepts d'héritages, de classes abstraites et de templates ;
%     \item l'existance de designs de programmation comme celui du singleton ;
%     \item des structures de données comme les containeurs et les listes chaînées.
% \end{itemize}
% On pourra continuer d'explorer le C++ avec les expressions lambda \cite{lambda}, ou la différence entre le polymorphisme statique et dynamique. La lecture de livres spécialisés est recommandée \cite{gamma1995design}, \cite{jumpingallain}.




%%=============================================================================
%%=============================================================================
\ifstandalone

    \appendix
    \printglossary[title=Vocabulaire et abbréviations]
	\bibliographystyle{IEEEtran}
	\bibliography{\rootPath Annexes/biblio.bib}
\fi
%%=============================================================================
%%=============================================================================



\end{document}
