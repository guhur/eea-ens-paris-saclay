\documentclass[conference]{IEEEtran}
\newcommand*{\rootPath}{../}
\usepackage{url}
\usepackage[utf8]{inputenc}
\newcommand{\inlinecode}{\texttt}
\usepackage{microtype}
% \usepackage[colorlinks]{hyperref}

% math and cs
% \usepackage[lined,boxed,commentsnumbered]{algorithm2e}
% \usepackage{amsmath}
% \usepackage{amssymb}
% \usepackage{mathrsfs}

% style
% \usepackage{booktabs}
% \usepackage{multirow}
\usepackage{listings} % code highlighting
\lstset{language=C++,
                basicstyle=\ttfamily,
                keywordstyle=\color{blue}\ttfamily,
                stringstyle=\color{red}\ttfamily,
                commentstyle=\color{green}\ttfamily,
                morecomment=[l][\color{magenta}]{\#}
}
\usepackage{todonotes}
\usepackage{standalone}


% graph
\usepackage{graphicx}
\usepackage[outdir=./]{epstopdf}
\usepackage{array}



% \usepackage[french]{babel}
\usepackage[nomain,acronym,xindy,toc]{glossaries}
\makeglossaries
\usepackage[xindy]{imakeidx}
\makeindex




% questions
\newcounter{question}
\newcommand\Que[1]{%
    \bfseries
   \leavevmode\par
   \stepcounter{question}
   \noindent
   Q  --- #1\par\mdseries}


% reponses
\def\showanswers{1}% Set this =0 to hide, =1 to show
\newcommand\Ans[2][]{%
    \ifnum\showanswers=1
        \bfseries
        \leavevmode\par\noindent
       {\leftskip37pt
        R --- \textbf{#1}#2\par}\mdseries
    \fi
}


\newcommand\darwin{\texttt darwin-op}











\newglossaryentry{firmware}{
    name={firmware},
    description={programme intégré dans un matériel informatique (ordinateur, photocopieur, automate (API, APS), disque dur, routeur, appareil photo numérique, etc.) pour qu'il puisse fonctionne}
}


\newglossaryentry{framework}{
    name={framework},
    description={ensemble cohérent de composants logiciels structurels, qui sert à créer les fondations ainsi que les grandes lignes de tout ou d’une partie d'un logiciel (architecture)}
}
\newglossaryentry{instance}{
    name={instance},
    description={objet constituant un exemplaire de la classe}
}


\standalonetrue

\begin{document}

%%=============================================================================



%%=============================================================================
%-------------------------------------------------------------------------------
%   INTRODUCTION
%-------------------------------------------------------------------------------

\section{Introduction}
\label{sec:intro}


Chaque langage de programmation développe une propre philosophie, et chaque philosophie peut être appréhendée par les paradigmes de programmation.

Décrite par son fondateur Bjarne Stroustrup \cite{stroustrup}, la philosophie du C++ propose entre autres de :
\begin{itemize}
    \item laisser le choix d'un style de programmation aux développeurs ;
    \item apporter les options utiles est plus important que de veiller à ce qu'elle soit mal utilisée ;
    \item organiser les programmes de manière séparés et bien définies.
\end{itemize}
Cela signifie que le développeur a une grande liberté dans son utilisation du langage. Pour éviter de produire un code difficile à relire ou à étendre \cite{stroustrup2}, le développeur doit donc comprendre les paradigmes utilisés.

Le C++ utilise la programmation orientée objet et générique. C'est une extension de la programmation procédurale (utilisée en C par exemple). Dans cette dernière, des fonctions sont utilisées, ce qui permet d'éviter des redondances de code et de mettre en place des algorithmes récursifs.
La programmation orientée objet et générique utilise les notions de classes et de modèles pour mutualiser et unifier les informations entre différentes entitées.


Ce travail pratique va explorer ces notions au travers d'exemples de difficultés progressives sur le robot darwin-ops.
Dans une première partie, le robot darwin-op et son framework sont explorés.
Dans une seconde partie, un module de mouvement (MotionModule) est intégré.

Le texte du TP et le code utilisé sont dispobles sur le répertoire git \cite{git}. En cas d'erreurs ou de mises-à-jour, merci de lancer un pull-request.

\end{document}
