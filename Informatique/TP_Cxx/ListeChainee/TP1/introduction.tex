\documentclass[conference]{IEEEtran}
\newcommand*{\rootPath}{../}
\usepackage{url}
\usepackage[utf8]{inputenc}
\usepackage{tabulary}
\usepackage{tabularx}
\usepackage{listings} % code highlighting
\lstset{language=Ruby,
                basicstyle=\ttfamily,
                keywordstyle=\color{blue}\ttfamily,
                stringstyle=\color{red}\ttfamily,
                commentstyle=\color{green}\ttfamily,
                morecomment=[l][\color{magenta}]{\#}
}


% graph
\usepackage{graphicx}
\usepackage[outdir=./]{epstopdf}
\usepackage{array}



% \usepackage[french]{babel}


\standalonetrue

\begin{document}

%%=============================================================================



%%=============================================================================
%-------------------------------------------------------------------------------
%   INTRODUCTION
%-------------------------------------------------------------------------------

\section{Introduction}
\label{sec:intro}

\subsection{Présentation du C++}

Chaque langage de programmation développe une propre philosophie, et chaque philosophie peut être appréhendée par les paradigmes de programmation.

Décrite par son fondateur Bjarne Stroustrup \cite{stroustrup}, la philosophie du C++ propose entre autres de :
\begin{itemize}
    \item laisser le choix d'un style de programmation aux développeurs ;
    \item apporter les options utiles est plus important que de veiller à ce qu'elle soit mal utilisée ;
    \item organiser les programmes de manière séparés et bien définies.
\end{itemize}
Cela signifie que le développeur a une grande liberté dans son utilisation du langage. Pour éviter de produire un code difficile à relire ou à étendre \cite{stroustrup2}, le développeur doit donc comprendre les paradigmes utilisés.

Le C++ utilise la programmation orientée objet et générique. C'est une extension de la programmation procédurale (utilisée en C par exemple). Dans cette dernière, des fonctions sont utilisées, ce qui permet d'éviter des redondances de code et de mettre en place des algorithmes récursifs.
La programmation orientée objet et générique utilise les notions de classes et de modèles pour mutualiser et unifier les informations entre différentes entitées.


\subsection{Listes chaînées}

L'objectif de ce TP est d'implémenter une liste chaînée.
Le fonctionnement de la liste chaînée est que chaque élément possède, en plus de la donnée, un pointeur vers un élément qui lui est contigu dans la liste.

Par rapport aux tableaux, les listes chaînées permettent d'ajouter ou de supprimer dynamiquement des éléments. De plus, elles n'utilisent pas de mémoire continue.
L'usage d'une liste est donc souvent préconisé pour des raisons de rapidité de traitement, lorsque l'ordre des éléments est important et que les insertions et les suppressions d'éléments quelconques sont plus fréquentes que les accès.
En effet, les insertions en début ou fin de liste et les suppressions se font en temps constant car elles ne demandent au maximum que deux écritures.

En revanche, l'accès à un élément quelconque nécessite le parcours de la liste depuis le début jusqu'à l'index de l'élément choisi. L'accès est donc plus long que les tableaux. De plus, les listes chaînées utilisent plus de mémoire.

Par exemple, il est préfèrable d'utiliser un tableau pour stocker une image car toute la taille est bien connue dès la mise en mémoire, alors que l'ensemble des utilisateurs connectés seront mieux stockés dans une liste chaînée.



Le texte du TP et le code utilisé sont disponibles sur le répertoire git \cite{git}. En cas d'erreurs ou de mises-à-jour, merci de lancer un \emph{pull-request} ou une \emph{issue}.

\end{document}
