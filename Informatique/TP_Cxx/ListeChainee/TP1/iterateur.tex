\documentclass[abstracton]{scrartcl}
\newcommand*{\rootPath}{../}
\usepackage{url}
\usepackage[utf8]{inputenc}
\newcommand{\inlinecode}{\texttt}
\usepackage{microtype}
% \usepackage[colorlinks]{hyperref}

% math and cs
% \usepackage[lined,boxed,commentsnumbered]{algorithm2e}
% \usepackage{amsmath}
% \usepackage{amssymb}
% \usepackage{mathrsfs}

% style
% \usepackage{booktabs}
% \usepackage{multirow}
\usepackage{listings} % code highlighting
\lstset{language=C++,
                basicstyle=\ttfamily,
                keywordstyle=\color{blue}\ttfamily,
                stringstyle=\color{red}\ttfamily,
                commentstyle=\color{green}\ttfamily,
                morecomment=[l][\color{magenta}]{\#}
}
\usepackage{todonotes}
\usepackage{standalone}


% graph
\usepackage{graphicx}
\usepackage[outdir=./]{epstopdf}
\usepackage{array}



% \usepackage[french]{babel}
\usepackage[nomain,acronym,xindy,toc]{glossaries}
\makeglossaries
\usepackage[xindy]{imakeidx}
\makeindex




% questions
\newcounter{question}
\newcommand\Que[1]{%
    \bfseries
   \leavevmode\par
   \stepcounter{question}
   \noindent
   Q  --- #1\par\mdseries}


% reponses
\def\showanswers{1}% Set this =0 to hide, =1 to show
\newcommand\Ans[2][]{%
    \ifnum\showanswers=1
        \bfseries
        \leavevmode\par\noindent
       {\leftskip37pt
        R --- \textbf{#1}#2\par}\mdseries
    \fi
}


\newcommand\darwin{\texttt darwin-op}











\newglossaryentry{firmware}{
    name={firmware},
    description={programme intégré dans un matériel informatique (ordinateur, photocopieur, automate (API, APS), disque dur, routeur, appareil photo numérique, etc.) pour qu'il puisse fonctionne}
}


\newglossaryentry{framework}{
    name={framework},
    description={ensemble cohérent de composants logiciels structurels, qui sert à créer les fondations ainsi que les grandes lignes de tout ou d’une partie d'un logiciel (architecture)}
}
\newglossaryentry{instance}{
    name={instance},
    description={objet constituant un exemplaire de la classe}
}


\standalonetrue

\begin{document}
\lstset{language=C++}
%%=============================================================================



%%=============================================================================
%-------------------------------------------------------------------------------
%   Iterateur
%-------------------------------------------------------------------------------

\section{Programmation d'un itérateur}
\label{sec:exploration}



On propose de compléter cette liste chaînée afin d'en faire un itérateur. Pour cela, on créera une nouvelle classe \lstinline{Iterateur}, qui va gérer un curseur sur un nœud (dont le pointeur est stocké dans un attribut). Cette classe doit implémenter les méthodes suivantes :
\begin{itemize}
    \item \lstinline{void operator++();}
	\item \lstinline{bool operator!=(Iterateur rval);}
	\item \lstinline{bool operator==(Iterateur rval);}
	\item \lstinline{int operator*();}
\end{itemize}



\begin{figure}
	\centering
	\includegraphics[width=4in]{\rootPath diagramme-classes-2.png}
	\caption{Diagramme des classes}
	\label{fig:d-c2}
\end{figure}


\Que{Faire apparaître les relations entre les classes dans la figure \ref{fig:d-c2}}

\Ans{
\begin{figure}
	\centering
	\includegraphics[width=4in]{\rootPath diagramme-classes-3.png}
	\caption{Diagramme des classes}
	\label{fig:d-c2}
\end{figure}
}

\Que{Quelle classe est désormais amie avec quelle classe ?}
\Ans{ListeChainee est amie avec Iterateur et Noeud. Iterateur est amie avec Noeud.}


\Que{Implémentez l'itérateur des nœuds. Proposez un exemple de boucle \lstinline{for} pour parcourir la liste chaînée.}

\Ans{\lstinputlisting[language=C++, firstline=1, lastline=73]{\rootPath Code/ListeChaineeIterateur.cpp}}


\end{document}
