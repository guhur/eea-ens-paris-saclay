\documentclass[abstracton]{scrartcl}
\newcommand*{\rootPath}{../}
\usepackage{url}
\usepackage[utf8]{inputenc}
\usepackage{tabulary}
\usepackage{tabularx}
\usepackage{listings} % code highlighting
\lstset{language=Ruby,
                basicstyle=\ttfamily,
                keywordstyle=\color{blue}\ttfamily,
                stringstyle=\color{red}\ttfamily,
                commentstyle=\color{green}\ttfamily,
                morecomment=[l][\color{magenta}]{\#}
}


% graph
\usepackage{graphicx}
\usepackage[outdir=./]{epstopdf}
\usepackage{array}



% \usepackage[french]{babel}


\standalonetrue

\begin{document}
\lstset{language=C++}
%%=============================================================================



%%=============================================================================
%-------------------------------------------------------------------------------
%   DIAGRAMME CLASSES
%-------------------------------------------------------------------------------

\section{Diagramme des classes}
\label{sec:exploration}


\begin{figure}
	\centering
	\includegraphics[width=4in]{\rootPath diagramme-classes}
	\caption{Diagramme des classes}
\end{figure}


La figure \ref{fig:diag-classes} représente le diagramme des classes d'une liste chaînée. % ("\$\*" est une erreur pour \*).
Deux classes sont représentées :
\begin{itemize}
    \item ListeChainee contient la chaîne et permet d'ajouter des éléments à la chaîne.
    \item Noeud est un maillon de la chaîne.
\end{itemize}

\Que{Quelle est la relation entre les deux classes ? Quelle est la différence entre un + et un - ? Donnez une description de chaque méthode et de chaque attribut.}

\Ans{Classe amie. + pour public, - pour privé. \_valeur: valeur à stocker. \_suivant: prochain maillon de la chaîne. Constructeurs et destructeurs. push\_back: ajoute un maillon à la fin de la chaîne. push\_front: ajoute un maillon au début. get\_value: valeur du n-ième Noeud}


\Que{Pourquoi est-ce important d'implémenter un destructeur ?}

\Ans{Relâcher les ressources.}


%
\Que{Implémentez ce diagramme de classes et proposez un exemple d'utilisation.}

\Ans{\lstinputlisting[language=C++, firstline=1, lastline=73]{\rootPath Code/ListeChaineeDebut.cpp}}


\end{document}
